% !TEX TS-program = pdflatex
% !TEX encoding = UTF-8 Unicode


\documentclass[review, endfloat]{elsarticle}

\usepackage[utf8]{inputenc}

\usepackage{graphicx}
\usepackage{booktabs}
%\usepackage{doi}
\usepackage{hyperref}

\title{Adverse events following live attenuated herpes zoster vaccine: a self-controlled case series analysis}

% Authors and affiliations

\author[1]{James Totterdell\corref{cor1}}
\ead{jamestotterdell@telethonkids.org.au}

\author[2]{Anastasia Phillips}
\ead{cvr@sayahna.org}

\author[3]{Catherine Glover}
\ead[url]{www.stmdocs.in}

\cortext[cor1]{Corresponding author}
\address[1]{Wesfarmers Centre for Vaccines and Infectious Diseases, Telethon Kids Institute}
\address[2]{affil}
\address[3]{affiliation}


\begin{document}

\begin{frontmatter}

\begin{abstract}

\emph{Objective}: To investigate the risk of pre-specified adverse events following live attenuated herpes zoster vaccine (ZVL) in older adults attending primary care providers in Australia.

\emph{Methods}: Individuals aged 70 to 79 years who received ZVL between 1 November 2016 and 31 July 2018 were identified within a nationally representative primary care database. The self-controlled case series (SCCS) method was used to estimate the seasonally-adjusted relative incidence (RI) of seven outcome events (injection site reaction (ISR), burn [negative control], myocardial infarction (MI), stroke, any rash, rash with a prescription for an antiviral within 2 days of the rash-related encounter, and clinical attendance) during a post-vaccination, at-risk window compared with a time distant to vaccination. Sensitivity analyses examined the effect of concomitant vaccination (influenza and 23-valent pneumococcal vaccination) and restriction to first outcome event. 

\emph{Results}: A total of 332,988 vaccination encounters in 150,054 individuals were identified during the study period. The most common outcome identified was clinical attendance (\textgreater 2 million events) followed by rash (12,309 events); ISR was the rarest outcome (177 events). There was an increased RI of ISR in the 7 days following ZVL (RI = 77.4; 95\% CI = [48.1, 124.6]). No change to the RI of MI (0.74; [0.41, 1.33]), rash (0.97; [0.8 to 1.08]), or rash with antiviral (0.83; [0.62 to 1.10]) were identified in the 42 days following ZVL. The RI of clinical attendance (0.94; [0.94, 0.95]) and stroke (0.6; [0.4, 0.8]) were lower in the 42 days following administration of ZVL.

\emph{Conclusions}: No new safety concerns were identified for ZVL in this study based on a novel primary care data source using an SCCS design. An expected increased risk of ISR was identified; findings in relation to cardiovascular disease were reassuring.

\end{abstract}

\begin{keyword}
vaccine \sep self-controlled case series
\end{keyword}

\end{frontmatter}


\section{Introduction}

Herpes zoster (HZ) is a localised, painful, vesicular skin rash resulting from reactivation of varicella-zoster virus. The average lifetime risk is around 30\% but increases with age \citep{brisson2001epidemiology}. Prior to implementation of vaccination programs, the incidence of HZ in Australia was reported to be 10 per 1000 persons aged 50 years and older \citep{stein2009herpes}, similar to rates observed in Europe \citep{pinchinat2013similar} and the United States (US) \citep{insinga2005incidence}. The risk of post-herpetic neuralgia, a chronic neuropathic pain syndrome common following HZ, also increases with age. Disseminated disease can occur in people who are immunosuppressed. 

Live attenuated varicella-zoster vaccine (ZVL) was registered for use in Australia in 2006 and available from 2014 for people aged over 50 years \citep{yawn2007population}. It is recommended for immunocompetent adults over 60 years of age. In November 2016, a funded ZVL immunisation program commenced under the Australian National Immunisation Program (NIP) for adults aged 70 years, with catch-up for those aged 71–79 years funded until October 2021. 

ZVL was evaluated in large, pre-licensure clinical trials with no increased risk of serious adverse events (SAE), hospitalized adverse events or death \cite{oxman2005, gagliardi2016vaccines, schmader2012}. The rate of injection site reactions (ISR) was higher in vaccine than placebo groups. A post-licensure clinical trial demonstrated a similar safety profile, with no statistically significant difference in the rate of SAE up to 182 days following vaccination \citep{murray2011}. Post-licensure passive surveillance has been consistent with clinical trials. Of 23,556 reports submitted by healthcare providers to the Merck, Sharp, \& Dohme Corp (MSD) Adverse Event (AE) global safety database between May 2006 and May 2016, 93\% of reports were non-serious, with ISR the most commonly reported AE (20.5\%), followed by HZ (8.6\%) and rash (4.2\%) \citep{willis2017herpes}. Of 23,092 reports submitted to the US Vaccine Adverse Events Monitoring System (VAERS) from May 2006 to January 2015, 96\% were classified as non-serious, with ISR, HZ and rash the most frequently reported non-serious AE \citep{miller2018post}.

Post-marketing surveillance methods are limited by the potential for biased reporting and, for ZVL, is confounded by the higher prevalence of chronic disease in the older target population \citep{miller2018post}. The self-controlled case series (SCCS) method was developed for vaccine safety assessment and has the potential to automatically adjust for unmeasured time-invariant confounders by allowing individuals to act as their own control \citep{petersen2016}. SCCS methodology has previously been used to examine ZVL using US data from managed care cohorts \citep{tseng2012} and administrative claims data \citep{minassian2015}. In Australia, older patients typically receive ZVL from their private general practitioner (GP). While practices maintain their own electronic patient records, the Australian Government Department of Health funds NPS MedicineWise to collate GP data through the MedicineInsight program. This study used nationally representative GP data from MedicineInsight to examine the risk of pre-specified AE in the target NIP cohort.

\section{Methods}

\subsection{Study setting}

The MedicineInsight data set consists of longitudinal, de-identified, whole-of-practice data extracted from the clinical information systems (CIS) of participating practice sites across Australia. It includes between 15\% and 20\% of the Australian population.  Data is extracted on patient demographics, practice encounters (excluding progress notes), diagnoses, vaccinations, prescriptions, pathology tests and referrals. Practice encounters can include clinical (a medical or nursing appointment) or non-clinical (an entry by administration staff) encounters. Within-site individual identifiers are used to identify records common to an individual attending that site. 

\subsection{Study population}

The target population was individuals aged 70-79 years who were eligible to receive ZVL, 23-valent pneumococcal (23vPPV), or seasonal inactivated influenza vaccine. Although the primary vaccine of interest was ZVL, individuals who had received 23vPPV and seasonal inactivated influenza vaccines were also included for two reasons: ZVL may commonly be co-administered with these two vaccines meaning that any events identified might be attributable to these other vaccines; and to estimate the relative incidence of adverse events in other vaccines using the same data source and methods as a comparator for the ZVL estimates.  All MedicineInsight records were obtained for individuals 70–79 years of age at the date of their vaccination who received ZVL, 23vPPV or influenza vaccine(s) between 1 November 2016 (when the funded ZVL program began) to 31 July 2018. We treated this study population as a random sample of the target population. 

\subsection{Study design}

The SCCS design was developed for vaccine safety evaluation \citep{farrington1995} and has been used in a variety of settings \citep{buttery2011intussusception, bakken2015febrile, stowe2016risk}. The method estimates the relative incidence of an outcome event within a risk window following exposure (i.e. vaccination) compared to the incidence at all other times under observation. Only individuals who have experienced an outcome event contribute to the relative incidence estimate and the design potentially controls for time-invariant confounders.

This study investigated the relative incidence of seven pre-specified outcome events (ISR [positive control], burn [negative control], myocardial infarction, stroke, any rash, rash with a prescription for an antiviral within 2 days of the rash-related encounter, and clinical attendance) in a post-vaccination, at-risk window, with the incidence of these outcome events at a time distant to vaccination. ISR was included as a positive control given consistent evidence of an increased risk of ISR in pre-licensure and post-licensure studies. Burn was included as a negative control with any events considered unrelated to vaccination.

Individuals in the study population may not have been at risk for events throughout the entirety of the study period. Therefore, we defined an individual’s observation period in terms of their historical activity at the site and year of death information. An individual’s start date of observation was defined as the latest of 1 November 2016, or 365 days after their first recorded activity at the site (any encounter, diagnosis, or prescription). An individual’s end date of observation was defined as 31 December in the year prior to their death for individuals who had year of death recorded, and 31 July 2018 for individuals who had no year of death recorded. The lead time of 365 days from an individual’s start of site activity was specified to allow for a catch up in medical information. Therefore, the maximum observation period allowed was 638 days. 

Exposure (vaccination) was defined as an immunisation record for any of the three vaccines under study with a date of administration occurring within the individual’s observation period. Due to the mix of clinically coded and free text entries, vaccination records for the study vaccines were identified via targeted, free-text search criteria. The date of vaccination was set as the administration date specified in the vaccination record. Individuals with multiple vaccination records for ZVL or 23vPPV during their observation period were excluded as these vaccines are generally recommended to be given as a single dose for older adults. We enforced a minimum time between influenza vaccinations of 126 days because a single dose is generally recommended each season. Any influenza records occurring within 126 days of an individual’s previous influenza vaccintation were excluded from analyses.

Except for clinical attendance, outcome events were identified using free-text regular expression searches of reason for encounter, reason for diagnosis and reason for prescription fields. For each record matching an outcome event, we matched encounters, diagnoses, and prescriptions on their respective dates to identify related events and then ordered all records by date of occurrence. Records of clinical attendance were identified as any site encounter record excluding those identified to be non-clinical (administrative). 

\subsection{Definition of risk windows}

Risk windows were defined based on biologically plausible windows supported by evidence. These were defined as 1–7 days post vaccination for ISR and 1–42 days post vaccination for all other outcome events. The basis for the length of the risk window for systemic adverse events was the 42 day window used in pre-licensure clinical trials and post-licensure studies. This time period is also biologically plausible for cardiovascular events, which have been observed following wild-type varicella-zoster virus, particularly 1 to 4 weeks following infection with viral replication in arterial walls the proposed mechanism for cerebrovascular disease. This risk window is also biologically plausible for rash, with varicella-like rash after 6 weeks more likely to represent wild-type infection. The risk windows for burn (the negative control) and clinical attendance were chosen to be consistent with the risk window for systemic events. For ISR, the risk window was based on the short median time to ISR (~2 days) in the Shingles Prevention Study (SPS) and post-licensure surveillance and the identification of a signal for cellulitis within 7 days in another post-licensure SCCS.

\subsection{Statistical methods}

Relative incidence estimates were obtained from an SCCS model \citep{farrington2018} using the windows previously defined. Given that the study period spanned 1 November 2016 to 31 July 2018, we additionally specified fixed windows to adjusted for seasonal effects via cut-points for: 1 December, 1 March, 1 June, and 1 September. The main analysis modelled all exposures (vaccines) together and allowed for recurrent events for each outcome (i.e. every outcome event occurring during the observation period could contribute towards the relative incidence estimate). 

Dependence between recurrent events was explored visually by looking at Nelson-Aalen estimators for the cumulative hazard and hisotgrams of gap times. Dependent events violates the Poisson assumption of the SCCS model. For events where the risk of an outcome event was not considered independent of the first occurrence, sensitivity analyses were undertaken which only included the first outcome event observed.  Sensitivity analyses assessing each vaccine independently, excluding co-administered vaccines from the analysis, were also undertaken. 

Analyses were conducted using R version 3.5.1 \citep{rmanual} and the gnm package version 1.1-0 \citep{rgnm}.

\subsection{Ethical approval}

This MedicineInsight program was approved through the Royal Australian College of General Practitioners National Research and Evaluation Ethics Committee (NREEC) in December 2017 (NREEC 17-017). Approval for use of MedicineInsight data in this study was received from the NPS MedicineWise external Data Governance Committee on 23 November 2016 and an amended version on 29 September 2017.This study was also approved by the Sydney Children’s Hospitals Network Human Research Ethics Committee (HREC/17/SCHN/159). 

\section{Results}

A total of 337,294 vaccination records for 150,756 individuals from 456 MedicineInsight sites were obtained. These include sites in major cities and in rural and remote areas, similar to the distribution of the Australian population over these areas. After excluding individuals with multiple ZVL or 23vPPV vaccinations, and excluding multiple influenza vaccinations within 126 days of each other, a total of 332,988 vaccinations for 150,054 individuals were included in the analysis. 

\subsection{Population, outcome, and exposure characteristics}

In general, the number of vaccination records declined with age with ZVL increasing slightly at 79 years of age. About 82\% of zoster and 89\% of influenza vaccinations were administered alone, while 47\% of pneumococcal vaccinations were administered alone and 43\% were administered with influenza vaccine.

The most common outcome event identified was clinical attendance, with over 2 million attendances observed among exposed individuals during the study period. The next most commonly identified outcome event was any rash, with 12,309 events observed. The rarest outcome event identified was injection site reaction, with 177 events observed.

\subsection{Self-controlled case series analysis}

\subsubsection{Injection site reactions}

\section{Discussion}

\bibliographystyle{elsarticle-num-names}
\bibliography{ausvax-paper}

\end{document}
